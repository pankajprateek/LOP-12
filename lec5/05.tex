\documentclass{beamer}  
\usepackage{amsmath}
\usepackage{graphicx}
\usepackage{url}
\usepackage{listings}
\usepackage{fancyvrb}
\usepackage[T1]{fontenc}
\usepackage{hyperref}
\mode<presentation>
{ \usetheme{Warsaw} }
\title{Graphs
}

\author{League of Programmers}
\institute{ACA, IIT Kanpur}
\date
 
\AtBeginSection[]  % "Beamer, do the following at the start of every section"
{
\begin{frame}<beamer> 
\frametitle{Outline} % make a frame titled "Outline"
\tableofcontents[currentsection]  % show TOC and highlight current section
\end{frame}
}

\begin{document}
%----------- titlepage ----------------------------------------------%
\begin{frame}
  \titlepage
\end{frame}

\section{Basics}

\begin{frame}[<+->]{Basics}
\begin{block}{What are Graphs?}
  \begin{itemize}
    \item An abstract way of representing connectivity using nodes (or vertices) and edges
    \item Graph G=(V,E) , $E \subseteq V*V$
    \item We will label the nodes from 1 to n (V)
    \item m edges connect some pairs of nodes (E)
    \item Edges can be either one-directional (directed) or bidirectional
    \item Nodes and edges can have some auxiliary information
  \end{itemize}
\end{block}
\end{frame}

\begin{frame}[<+->]{Basics}
\begin{block}{Terminologies}
  \begin{itemize}
    \item Vertex: node of a graph
    \item Adjacency(u) = $\{v | (u, v) \in E\}$
    \item Degree(u) = |Adjacency(u)|
    \item Subgraph: A subset of vertices and edges is a subgraph
    \item Walk: A sequence $v_1$, $v_2$, \dots, $v_k$ such that $(v_i, v_{i+1}) \in E$
    \item Trial: A trial is a walk in which no edge occurs twice
    \item Closed path: A walk where starting and ending vertex are the same
  \end{itemize}
\end{block}
\end{frame}

\begin{frame}[<+->]{Basics}
\begin{block}{Storing Graphs}
  \begin{itemize}
    \item We need to store both the set of nodes V and the set of edges E
    \item Nodes can be stored in an array. Edges must be stored in some other way.
    \item We want to support the following operations\\
      \begin{itemize}
	\item Retrieving all edges incident to a particular node
	\item Testing if given two nodes are directly connected
      \end{itemize}
  \end{itemize}
\end{block}
\end{frame}

\begin{frame}[<+->]{Basics}
\begin{block}{Adjacency Matrix}
  \begin{itemize}
    \item An easy way to store connectivity information
    \item Checking if two nodes are directly connected: O(1) time
    \item Make an n x n matrix A
    \item a[i][j] = 1 if there is an edge from i to j
    \item a[i][j] = 0 otherwise
    \item Uses O($n^2$) memory. So, use when n is less than a few thousands, AND when the graph is dense
  \end{itemize}
\end{block}
\end{frame}

\begin{frame}[<+->]{Basics}
\begin{block}{Adjacency List}
  \begin{itemize}
    \item Each vertex maintains a list of vertices that are adjacent to it.
    \item Lists have variable lengths
    \item We can use: vector< vector<int> >
    \item Space usage: O(n + m)
    \item Checking if edge (Vi,Vj) is present in G: \pause O(min(deg(Vi),deg(Vj)))
  \end{itemize}
\end{block}
\end{frame}

\begin{frame}[<+->]{Basics}
\begin{block}{Special Graphs}
  \begin{itemize}
    \item \small{{\bf Implicit graphs}\\
      Two squares on an 8x8 chessboard. Determine the shortest sequence of knight moves from one square to the other.}
    \item \small{{\bf Tree}: a connected acyclic graph\\
      The most important type of graph in CS\\
      Alternate definitions (all are equivalent!)\\}
      \begin{itemize}
	\item \small{connected graph with n-1 edges}
	\item \small{An acyclic graph with n-1 edges}
	\item \small{There is exactly one path between every pair of nodes}
	\item \small{An acyclic graph but adding any edge results in a cycle}
	\item \small{A connected graph but removing any edge disconnects it}
      \end{itemize}
    \end{itemize}
\end{block}
\end{frame}

\begin{frame}[<+->]{Basics}
\begin{block}{Special Graphs}
  \begin{itemize}
    \item \small{{\bf Directed Acyclic Graph (DAG)}}
    \item \small{{\bf Bipartite Graph}\\
      Nodes can be separated into two groups S and T such that edges exist between S and T only (no edges within S or within T)}
  \end{itemize}
\end{block}
\end{frame}

\section{Graph Traversal}
\begin{frame}[<+->]{Traversal}
\begin{block}{Graph Traversal}
  \begin{itemize}
    \item The most basic graph algorithm that visits nodes of a graph in certain order
    \item Used as a subroutine in many other algorithms
    \item We will cover two algorithms\\
      \begin{itemize}
	\item Depth-First Search (DFS): uses recursion
	\item Breadth-First Search (BFS): uses queue
      \end{itemize}
  \end{itemize}
\end{block}
\end{frame}

\begin{frame}[<+->]{Traversal}
\begin{block}{DFS}
  \begin{itemize}
    \item DFS(v): visits all the nodes reachable from v in depth-first order\\
      \begin{itemize}
      \item Mark v as visited
      \item For each edge $v \rightarrow u$:\\
	\hspace{3mm} If u is not visited, call DFS(u)
      \end{itemize}
    \item Use non-recursive version if recursion depth is too big (over a few thousands)
    \item Replace recursive calls with a stack
    \item Complexity\\
      \pause\hspace{2mm} Time: O(|V|+|E|)\\
      \pause\hspace{2mm} Space: O(|V|) [to maintain the vertices visited till now]
  \end{itemize}
\end{block}
\end{frame}

\begin{frame}[<+->]{Traversal}
\begin{block}{DFS: Uses}
  \begin{itemize}
    \item Biconnected components
    \item A node in a connected graph is called an articulation point if the deletion of that node disconnects the graph.
    \item A connected graph is called biconnected if it has no articulation points. That is, the deletion of any single node leaves the graph connected.
  \end{itemize}
\end{block}
\end{frame}

\begin{frame}[<+->]{Traversal}
\begin{block}{BFS}
  BFS(v): visits all the nodes reachable from v in breadth-first order
  \begin{itemize}
    \item Initialize a queue Q
    \item Mark v as visited and push it to Q
    \item While Q is not empty:\\
      \hspace{3mm} Take the front element of Q and call it w\\
      \hspace{3mm} For each edge $w \rightarrow u$:\\
	\hspace{6mm} If u is not visited, mark it as visited and push it to Q
    \item Same Time and Space Complexity as DFS
  \end{itemize}
\end{block}
\end{frame}

\begin{frame}[<+->]{Traversal}
\begin{block}{BFS: Uses}
  \begin{itemize}
    \item Finding a Path with Minimum Number of edges from starting vertex to any other vertex.
    \item Solve Shortest Path problem in unweighted graphs
    \item Spoj Problem\\ \url{http://www.spoj.pl/problems/PPATH/}
  \end{itemize}
\end{block}
\end{frame}

\begin{frame}[<+->]{Topological Sort}
\begin{block}{}
  \begin{itemize}
    \item Input: a DAG G = V, E
    \item Output: an ordering of nodes such that for each edge $u \rightarrow v$, u comes before v. There can be many answers
    \item Pseudocode\\
      \begin{itemize} 
	\item Precompute the number of incoming edges deg(v) for each node v
	\item Put all nodes with zero degree into a queue Q
	\item Repeat until Q becomes empty:\\
	  \hspace{3mm} Take v from Q\\
	  \hspace{3mm} For each edge $v \rightarrow u$\\
	    \hspace{6mm} Decrement deg(u) (essentially removing the edge $v \rightarrow u$)\\
	    \hspace{6mm} If deg u becomes zero, push u to Q
	\end{itemize}
    \item Time complexity: $O(n+m)$
  \end{itemize}
\end{block}
\end{frame}

\section{MST}
\begin{frame}[<+->]{Minimum Spanning Tree}
\begin{block}{}
  \begin{itemize}
    \item Input: An undirected weighted graph G = V, E
    \item Output: A subset of E with the minimum total weight that connects all the nodes into a tree
    \item There are two algorithms:
      \begin{itemize}
	\item Kruskal's algorithm
	\item Prim's algorithm
      \end{itemize}
  \end{itemize}
\end{block}
\end{frame}

\begin{frame}[<+->]{Kruskal's Algorithm}
\begin{block}{}
  \begin{itemize}
    \item Main idea: the edge e with the smallest weight has to be in the MST
    \item Keep different supernodes, which are “local MST’s” and then join them by adding edges to form the MST for the whole graph
    \item Pseudocode:\\
      \tt{\hspace{3mm} Sort the edges in increasing order of weight\\
      \hspace{3mm} Repeat until there is one supernode left:\\
	\hspace{6mm} Take the minimum weight edge e*\\
	\hspace{6mm} If e* connects two different supernodes:\\
	  \hspace{9mm} Connect them and merge the supernodes\\
	\hspace{6mm} Otherwise,\\
	  \hspace{9mm} ignore e*}
  \end{itemize}
\end{block}
\end{frame}

\begin{frame}[<+->]{Prim's Algorithm}
\begin{block}{Prim's Algo}
  Reading Homework
\end{block}
\end{frame}

\section{Shortest Path Algos}

\begin{frame}[<+->]{Floyd-Warshall Algorithm}
\begin{block}{}
  \begin{itemize}
    \item All pair shortest distance
    \item Runs in O($n^3$) time
    \item Algorithm\\
    \begin{itemize}
      \item Define f(i, j, k) as the shortest distance from i to j, using 1 ... k as intermediate nodes\\
	\begin{itemize}
	  \item f(i, j, n) is the shortest distance from i to j
	  \item f(i, j, 0) = cost(i, j)
	\end{itemize}
      \item The optimal path for f i, j, k may or may not have k as an intermediate node\\
	\begin{itemize}
	  \item If it does, f (i, j, k) = f (i,k k-1) + f(k, j, k-1)
	  \item Otherwise, f (i, j, k) = f (i, j, k-1)
	\end{itemize}
      \item Therefore, f (i, j, k) is the minimum of the two quantities above
      \end{itemize}
  \end{itemize}
\end{block}
\end{frame}

\begin{frame}[<+->]{Floyd-Warshall Algorithm}
\begin{block}{}
  \begin{itemize}
    \item Pseudocode:\\
    \tt{
      \hspace{2mm} Initialize D to the given cost matrix\\
      \hspace{2mm} For k = 1 \dots n:\\
      \hspace{2mm} For all i and j:\\
	  \hspace{6mm} $d_{ij} = min\{d_{ij}, d_{ik} + d_{kj}\}$
      }
      \item Can also be used to detect negative weight cycles in graph?\\
	\pause How?\\
	\pause If $d_{ij}+d_{ji} < 0$ for some i and j, then the graph has a negative weight cycle
  \end{itemize}
\end{block}
\end{frame}

\begin{frame}[<+->]{Dijkstra's Algorithm}
\begin{block}{}
  \begin{itemize}
    \item Used to solve Single source Shortest Path problem in Weighted Graphs
    \item Only for Graphs with positive edge weights.
    \item The algorithm finds the path with lowest cost (i.e. the shortest path) between that source vertex and every other vertex
    \item Greedy strategy
    \item Idea: Find the closest node to s, and then the second closest one, then the third, etc
  \end{itemize}
\end{block}
\end{frame}

\begin{frame}[<+->]{Dijkstra's Algorithm}
\begin{block}{}
  \begin{itemize}
    \item Pseudo code:\\
    \begin{itemize}
      \item Maintain a set of nodes S, the shortest distances to which are decided
      \item Also maintain a vector d, the shortest distance estimate from s
      \item Initially, S = s, and $d_v$ = cost(s, v)
      \item Repeat until S = V:\\
	\hspace{3mm} Find v $\notin$ S with the smallest $d_v$, and add it to S\\
	\hspace{3mm} For each edge $v \rightarrow u$ of cost c:\\
	  \hspace{8mm} $d_u = min\{d_u, d_v + c\}$
    \end{itemize}
  \end{itemize}
\end{block}
\end{frame}

\begin{frame}[<+->]{Dijkstra's Algorithm}
\begin{block}{}
  \begin{itemize}
    \item Time complexity depends on the implementation:\\
      \hspace{3mm} Can be O($n^2$ + m), O(m log n), O(n log n)
    \item Use priority\_queue<node> for implementing Dijkstra's
    \item SPOJ Problem\\
    \url{http://www.spoj.pl/problems/CHICAGO}
  \end{itemize}
\end{block}
\end{frame}

\begin{frame}[<+->]{Bellman-Ford Algorithm}
\begin{block}{}
  \begin{itemize}
    \item Single source shortest path for negative weights
    \item Can also be used to detect negative weight cycles
    \item Pseudo code:\\
      \begin{itemize}
	\item Initialize $d_s = 0$ and $d_v = \infty\ \forall v \neq s$
	\item For k = 1 \dots n-1:
	\item For each edge $u \rightarrow v$ of cost c:\\
	  \hspace{3mm} $d_v = min\{d_v , d_u + c\}$
      \end{itemize}
    \item Runs in O(nm) time
  \end{itemize}
\end{block}
\end{frame}

\section{Problems}

\begin{frame}{Problems}
Links:
\begin{enumerate}
\item \url{http://www.spoj.pl/problems/IOPC1201/}
\item \url{http://www.spoj.pl/problems/TRAFFICN/}
\item \url{http://www.spoj.pl/problems/PFDEP/}
\item \url{http://www.spoj.pl/problems/PRATA/}
\item \url{http://www.spoj.pl/problems/ONEZERO/}
\item \url{http://www.spoj.pl/problems/PPATH/}
\item \url{http://www.spoj.pl/problems/PARADOX/}
\item \url{http://www.spoj.pl/problems/HERDING/}
\item \url{http://www.spoj.pl/problems/PT07Z/}
\end{enumerate}
\end{frame}


\end{document}
